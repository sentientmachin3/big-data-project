\documentclass[a4paper]{article}
\usepackage[english]{babel}
%\usepackage[utf8]{inputenc}
\usepackage{amsmath}
\usepackage{graphicx}
\usepackage{float}
\usepackage{fixltx2e}
\usepackage{listings}
\usepackage{color}
\usepackage{textcomp}
\usepackage{latexsym}
\usepackage{lstautogobble}
\usepackage[colorinlistoftodos]{todonotes}
\usepackage[margin=3cm]{geometry}
\usepackage{hyperref}
\usepackage{fancyhdr}
\usepackage{tikz}
\usepackage{libertine}
\usepackage{csquotes}
\usepackage[backend=bibtex, style=numeric]{biblatex}
\addbibresource{biblio.bib}
\hypersetup{
	hidelinks, 
	colorlinks = true,
	linkcolor = black,
	citecolor = black
}

\pagestyle{fancy}
\lhead{\nouppercase{\leftmark}}
\rhead{\nouppercase{\rightmark}}
\chead{}
\lfoot{}
\cfoot{\thepage}
\rfoot{}
\renewcommand{\headrulewidth}{0.4pt}

\renewcommand{\ttdefault}{cmtt}
\begin{document}
	\clearpage
	\begin{titlepage}
		\centering
		\vspace*{\fill}
		{\scshape\LARGE Università degli Studi di Verona \par}
		\vspace{1.5cm}
		\line(1,0){240} \\
		{\huge\bfseries Authorship Attribution\par}
		\line(1,0){240} \\
		\vspace{0.5cm}
		{\scshape\Large Big Data project report\par}
		\vspace{2cm}
		{\Large\itshape Davide Bianchi VR424505\par
		\Large\itshape Matteo Danzi VR424987\par}
		\vspace{1cm}
		\vspace{5cm}
		\vspace*{\fill}
		% Bottom of the page
		{\large A.A. 2018/2019\par}
	\end{titlepage}
	\thispagestyle{empty}
	\newpage
	\tableofcontents
	\newpage
	
	\section{Introduction}
	The project aim was to design a tool which could establish the authorship of a manuscript by using specific criteria described later. 
	The used architecture is based on Hadoop, a distributed filesystem simulator, running in a docker container. 

	\section{Background and System Description}

	\noindent
	A container is a standard unit of software that packages up code and all its dependencies so the application runs quickly and reliably from one computing environment to another. A Docker container image is an executable package of software that includes everything needed to run an application: code, runtime, system tools, system libraries and settings.

	\bigskip

	\noindent
	Docker containers images become containers when they run on Docker Engine. They isolate software from its environment and ensure that it works uniformly despite differences for instance between development and staging.
	
	\bigskip

	\noindent
	The Cloudera\parencite{Cloudera} Docker image used in this project contains an Hadoop\parencite{Apache} Distributed File System (HDFS) partition. Cloudera provides a scalable, flexible, integrated platform that makes it easy to manage rapidly increasing volumes and
	varieties of data in an enterprise. It is distributed by Apache Hadoop. The Hadoop version used in this project is \texttt{Hadoop 2.6.0-cdh5.7.0}.

	\bigskip

	\noindent
	For the installation of Docker, the Cloudera Docker image and the execution of jobs using jar file it has been used the instructions of the course. The project is written using Java version \texttt{11.0.4}. For the code compilation we used IntelliJ version \texttt{2019.2.4} compiling using compatibility mode with version \texttt{1.7}.

	\section{Project Workflow}

	The first step the project is to analyze an amount of manuscripts, extracting relevant information from them in order to create a ``dictionary'' of known authors. Starting from these data, the program should take unknown manuscripts as input and establish the authorship by the comparing them with the ``dictionary'' previously created. 

	\newpage

	\printbibheading
	\printbibliography[nottype=book,heading=subbibliography,title={Online Sources}]





\end{document}